\chapter{Masse und Schwerpunktlage}
\section{Einführung}
Dieser Abschnitt enthält den Zuladungsbereich, innerhalb dessen das Segelflugzeug sicher betrieben werden darf.\\
Verfahren zum Wiegen des Segelflugzeuges und eine Beispielrechnung zur Ermittlung der zulässigen Beladegrenzen wird im Anhang aufgeführt.
\section{Definierte Massen, Hebelarme und Schwerpunktlagen}
\textbf{Massen}\\
\begin{tabular}{m{6,5cm} m{3cm}}

Maximale Flugmasse & $m_{max}=800kg$\\
Maximale Masse im Gepäckfach & $m_{G,max}=10kg$\\
Höchstzulässige Masse der nichttragenden Teile einschl. Zuladung & $m_{NT,max}=468kg$\\

\end{tabular}\\
\newline
\newline
\textbf{Zulässige Schwerpunktlagen}\\
\begin{tabular}{m{6,5cm} m{4cm}}
Vorderste Flugmassenschwerpunktlage & $x_{vorn}=245,3mm$\\
Hinterste Flugmassenschwerpunktlage & $x_{hinten}=428,6mm$\\

\end{tabular}\\
\newline
\newline
\textbf{Hebelarme}\\
\begin{tabular}{m{6,5cm} m{3cm}}
Piloten & $x_P=-445mm$\\
Gepäckfach & $x_G=200mm$\\

\end{tabular}\\
\newline
Die Bezugsebene ist die Vorderkante der Wurzelrippe.
\section{Wägebericht}
\begin{tiny}
\begin{tabular}{|m{1,8cm}|m{1,8cm}|m{2cm}|m{1,5cm}|m{1,5cm}|}
\hline
Datum & Leermasse [$kg$] & Leermassen- schwerpunkt [$mm$]  & Maximale Zuladung [$kg$] & Unterschrift\\

\hline

& & & &\\
14.03.12 & 579 & 552,9 & 221 & Hofmann\\
& & & &\\
\hline
& & & &\\
& & & &\\
& & & &\\
\hline
& & & &\\
& & & &\\
& & & &\\
\hline
& & & &\\
& & & &\\
& & & &\\
\hline
& & & &\\
& & & &\\
& & & &\\
\hline

\end{tabular}
\end{tiny}
%\section{Ausrüstungsverzeichnis}
%Stand: 09.03.2012\\
%\begin{tabular}{|l|l|l|l|l|}
%\hline
%Benennung & Baumuster & Hersteller & Einbauort\\
%\hline
%
%
%\end{tabular}