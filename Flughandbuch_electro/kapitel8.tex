\chapter{ Handhabung, Instandhaltung und Wartung}
\section{Einführung}
Dieser Abschnitt enthält die vom Hersteller empfohlenen Verfahren für einen angemessenen Umgang und die Instandhaltung des mit FES ausgestatteten Segelflugzeuges. Es enthält auch Inspektions- und Wartungsanweisungen, um die Leistung und die Zuverlässigkeit des Systems zu garantieren.

\section{Prüfintervalle}
Es liegt noch kein Wartungshandbuch für die B13 vor. \\
\newline
Die B13 muss regelmäßig einmal pro Jahr und nach größeren Reparaturen gewartet und nachgeprüft werden. \\
\newline
Dabei sollten mindestens folgende Wartungen durchgeführt werden:
\begin{itemize}
\item Gesamtes Flugzeug auf Risse, Löcher und Beulen untersuchen
\item Anschlussbeschläge auf einwandfreien Zustand (Spiel, Riefen, Korrosion) kontrollieren 
\item Metallteile (besonders der Steuerungsanlagen) auf Korrosion überprüfen, ggf. neu konservieren
\item Anschlüsse von Flügel und Leitwerke auf Spiel kontrollieren
\item Steuerung (besonders Bremsklappen) sind einer Funktionskontrolle zu unterziehen
\item Ruderausschläge und Anschlagpunkte der Steuerung nachprüfen
\item Fahrwerk und Schwerpunktkupplung kontrollieren 
\item Druckentnahmestellen der Druckanlagen auf Sauberkeit und die Leitungen auf Dichtigkeit überprüfen
\item Zustand und ordnungsgemäße Funktion aller Instrumente, Geräte und Ausrüstungsteile ist zu überprüfen
\end{itemize}

Zusätzlich sollten vor jedem Aufrüsten alle Anschlussbolzen und –buchsen sowie Leitwerks- und Flügelanschlüsse gereinigt und gefettet werden. \\
\newline
Da noch kein Schmierplan für die B13 vorliegt, sind die entsprechenden Lager des Öfteren zu kontrollieren und bei Bedarf nachzuschmieren. \\

\section{FES Wartungsintervalle}

Die Anweisungen zur Erhaltung der Lufttüchtigkeit im \textbf{FES WARTUNGSHANDBUCH} müssen eingehalten werden.

\section{Änderungen oder Reparaturen am Segelflugzeug}
Die verantwortliche Luftfahrtbehörde ist unbedingt \textbf{vor} jeder Änderung an dem Motorsegler zu unterrichten, um sicherzustellen, dass die Lufttüchtigkeit des Motorseglers nicht gefährdet wird. Erst nach Genehmigung der Änderungen von der Luftfahrtbehörde dürfen diese durchgeführt werden. \\
\newline
Größere Reparaturen sollten nur von fachkundigem Personal mit entsprechender Berechtigung durchgeführt werden. 
\newpage
\section{Handhabung am Boden/Straßentransport}

\subsection{Ziehen/Schieben}
Das Schleppen am Boden sollte über ein Seil mit einem Doppelring, welcher in der Schwerpunktkupplung eingehängt wird, erfolgen. Es sollte neben einer Person an der Fläche noch eine zweite Person in Nähe des Ausklinkknopfes den Schlepp begleiten. \\
\newline
Weiterhin ist für den Transport am Boden unbedingt der dafür vorgesehene Spornkuller zu verwenden. Die B13 hat zwei Spornkuller. Der größere Kuller enthält noch zusätzliche Auflageflächen zur Befestigung der Außenflächen während des Straßentransportes im Anhänger. Dieser Kuller eignet sich nicht für das Ziehen und Schieben am Boden. \\
Auf ausreichenden Luftdruck und festen Sitz ist bei dem normalen Kuller aufgrund der hohen Spornlast unbedingt zu achten.\\
\newline

\begin{color}{red}
\large{\underline{Warnung}}\\
Drücken, Ziehen oder Heben am Propeller oder am Spinner ist verboten!
\end{color}\\

\begin{color}{forestgreen}
\large{\underline{Wichtiger Hinweis}}\\
Zur Befestigung des Spornkullers sollte nicht auf die Nase gedrückt werden, da die Nase nur eine Abdeckung ist und keine tragende Wirkung hat.
\end{color}\\
\newline
Das Bewegen der B13 am Boden ohne Kuller sollte nur in Ausnahmefällen geschehen und ohne große Krafteinleitungen, die eine Bewegung um die Hochachse erzeugen, durchgeführt werden, damit der Sporn (insbesondere das Spornrad) und die Leitwerke nicht zu stark belastet werden.\\
\newline
Die Haube muss dabei in jedem Fall verriegelt werden. Es wird empfohlen, den vorderen Haubenspalt mit einem Mylarband, welches am Haubenrahmen befestigt wird, abzudichten, da es sonst während des Fluges zu unangenehmen Geräuschen kommen kann. 

\subsection{Abstellen und Lagern}
Die B13 sollte nur in gut belüfteten Räumen und Transportanhängern abgestellt und transportiert werden. Ein längeres Abstellen unter starker Sonneneinstrahlung oder Feuchtigkeit sollte möglichst vermieden werden, da es die Oberfläche deutlich schneller altern lässt.\\
\newline
Die Oberfläche (mindestens die Haube) sollte noch zusätzlich durch weiche, saubere Bezüge abgedeckt werden. \\

Ohne hochwertige Allwetterbezüge darf ein Segelflugzeug, das mit einem FES System ausgestattet ist, nicht bei Regen im Freien stehen. Der Motor und der Akkukasten müssen vor eindringendem Wasser geschützt werden. 

\subsection{Vorbereitung auf den Straßentransport}
Der Transport der B13 erfolgt in dem dafür vorgesehenen Transportanhänger. Vor dem Transport sollten unbedingt alle lockeren Gegenstände aus dem Cockpit entfernt und die nun losen Steuerstangen im Rumpf mit den dafür vorgesehenen Schonern bezogen werden. \\
\newline
Durch ihren breiten Rumpf wurde ein Befestigungssystem für die Außenflächen auf dem Rumpf vorgesehen. Um ein Loslösen der Außenflächen von der Halterung zu verhindern, ist auf eine vollständige Sicherung der Verschlüsse für die Halterungen hinter dem Cockpit und am Spornkuller zu achten. Auch der Rumpf und die Innenflächen sollten richtig in ihre Halterungen geschoben und anschließend gesichert werden. \\
\newline
Generell ist auf eine spannungsfreie Lagerung aller Einzelteile zu achten, da sich gerade bei hohen Temperaturen (wie sie in Transportanhängern auftreten können)  die einzelnen Flugzeugteile verziehen könnten. 

\section{Reinigung und Pflege}
Der Reinigung der Plexiglashaube sollte besondere Aufmerksamkeit geschenkt werden, da sie die freie Sicht der Piloten gewährleistet. Es ist unbedingt darauf zu achten, dass zum Säubern der Haube nur reichlich klares, sauberes Wasser und ein reines Ledertuch verwendet wird. Es sollte niemals trocken auf der Plexiglashaube gerieben werden. \\
\newline
Falls vorhanden, wird der Einsatz spezieller Reinigungsmittel für Plexiglashauben (z.B. Plexiklar) empfohlen. \\
\newline
Die Oberfläche der B13 sollte nach jedem Flugbetrieb mit einem weichen, sauberen Schwamm und viel klarem Wasser gereinigt werden. Zum Trocknen wird ein sauberes Ledertuch verwendet. \\
\newline
Das Putzen mit Wasser in der Nähe des Motorraumes und des Akkufachs sollte vermieden werden. Spinner und Propellerblätter sollten mit einem feuchten Schwamm oder einem weichen Baumwolltuch gereinigt werden.
In den Motorraum eingedrungene Feuchtigkeit ist sofort zu entfernen.\\
Klebebandreste können mit ein wenig Silikonentferner entfernt werden. Es sollte kein Aceton oder silikonhaltige Pflegemittel angewandt werden, da es die Lackschicht des Flugzeuges stark angreift oder den Aufwand bei Lackreparaturen deutlich erhöhen könnte. Weiterhin sollten Poliermittel und flüssiges Wachs zur Pflege der Oberfläche angewandt werden. \\
\newline
\begin{color}{forestgreen}
\large{\underline{Wichtiger Hinweis}}\\
Es ist unbedingt darauf zu achten, dass das Flugzeug vor Nässe geschützt wird. 	Eingedrungenes Wasser sollte schnellst möglichst entfernt werden. Dazu muss die B13 trocken gelagert und die abgerüsteten Flugzeugteile  öfters gewendet 	werden. 
\end{color}\\
\newline

Die Schwerpunktkupplung und das Hauptrad sind durch ihren Einbauort starken Verschmutzungen ausgesetzt (besonders nach Außenlandungen). Sie sollten daher laufend auf Verschmutzungen untersucht, gereinigt und geschmiert werden. 