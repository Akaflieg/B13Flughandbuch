\chapter{Notverfahren}
\pagecolor{red}
\section{Einführung}
Der vorliegende Abschnitt beinhaltet die Beschreibung der empfohlenen Verfahren bei eventuell eintretenden Notfällen.

\section{Abwerfen der Kabinenhaube}
Erfordert eine Situation das Abwerfen der Kabinenhaube, müssen folgende Schritte in der richtigen Reihenfolge ausgeführt werden:
\begin{itemize}
\item \textbf{Den roten Griff unten links auf dem Instrumentenbrett kräftig nach hinten bis zum Anschlag durchziehen}
\item \textbf{Haube nach oben wegstoßen}
\end{itemize}

Durch das Ziehen des roten Griffes am Instrumentenpilz wird die Haube an ihren seitlichen Befestigungen gelöst. Im vorderen Teil der Haubenmimik befinden sich zwei vorgespannte Federn, die nach dem Entriegeln die Haube vorne in die Strömung drücken. Die jetzt angreifenden Luftkräfte reißen die Haube nach hinten weg, wobei sie dabei eine definierte Drehung um die hintere Aufhängung (an der Gasdruckfeder) vollzieht. Diese Aufhängung ist mit einer Sollbruchstelle ausgestattet, die sich während oder unmittelbar nach der Drehung der Haube löst.\\
Falls nötig, muss die Haube zusätzlich mit beiden Händen nach oben weggedrückt werden.\\
\newline
\newline
\begin{color}{white}
\large{\underline{Wichtiger Hinweis}}\\
Bei ausgefahrenem Fahrwerk muss der Griff für den Haubennotabwurf leicht gedreht werden.
\end{color}\\

\begin{color}{white}
\large{\underline{Warnung}}\\
Vor dem Abwerfen der Kabinenhaube, wenn möglich den Motor
stoppen und das Antriebsystem ausschalten.
\end{color}

\section{Notausstieg}

Bei einem Notabsprung im Flug sollte man sich an die folgende Reihenfolge halten: 
\begin{enumerate}
\item \textbf{Haube} - abwerfen
\item \textbf{Gurtzeug} - öffnen
\item \textbf{Ausstieg} - mit beiden Armen über den Haubenrand hebeln (Körper möglichst anhocken) und dann vom Flugzeug abdrücken
\end{enumerate}

\begin{color}{white}
\large{\underline{Warnung}}\\
Vor dem Notausstieg, wenn möglich Motor stoppen und das Antriebsystem ausschalten.
\end{color}

\section{Beenden des überzogenen Flugzustands}
Der überzogene Flugzustand äußert sich bei einer Annäherung an die Mindestgeschwindigkeit (unabhängig von Wölbklappenstellung oder Querneigung) durch Weichwerden der Ruder, einer Taumelbewegung auf die eine Nickbewegung folgt, sowie Schütteln, Sackflug und Abreißerscheinungen am Rumpf.\\
\newline
\textbf{Dieser überzogene Flugzustand wird durch ein deutliches Nachlassen der Höhensteuerung und einer evt. Verminderung der Querneigung beendet.}\\
\newline
Wird im Sackflug der Anstellwinkel durch weiteres "`Ziehen"' deutlich erhöht, kann je nach Schwerpunktlage "`Trudeln"' die Folge eines einseitigen Abkippens über den Flügel sein.\\

\textbf{Im Motorbetrieb muss zuerst der Motor gestoppt werden. Danach wird der überzogene Flugzustand gem. Flughandbuch beendet}
\newpage
\section{Beenden des Trudelns}
Im Rahmen der Flugerprobung wurde das Trudeln mit unterschiedlichen Schwerpunktlagen, Drehrichtungen und Wölbklappenstellungen eingeleitet.\\
Bei den Wölbklappenstellungen \textbf{$-2$}, \textbf{$-1$} und \textbf{$0$} beträgt die maximale Fahrt beim Ausleiten \textbf{$190\frac{km}{h}$} und bei den Wölbklappenstellungen \textbf{$+1$} und \textbf{$+2$} beträgt sie \textbf{$170\frac{km}{h}$}.
\newline

\begin{color}{white} 
\underline{Warnung}\\
Bei der Wölbklappenstellung 'L' beträgt die zulässige Höchstgeschwindigkeit $130\frac{km}{h}$. Da diese Geschwindigkeit beim Ausleiten schnell erreicht werden kann, sollte man vor dem Ausleitvorgang eine andere Wölbklappenstellung rasten, um die Flugzeugstruktur nicht zu überlasten.\\
\end{color}


Um das Trudeln auszuleiten, kann bei der B13 die Standardmethode angewandt werden. Im Falle des Motorbetriebes, muss dieser zunächst gestoppt werden.

\begin{enumerate}
\item \textbf{Motor ggf. stoppen}
\item \textbf{Seitenruder gegen die Trudelrichtung}
\item \textbf{Höhenruder neutral stellen}
\item \textbf{Warten bis die Drehung aufhört}
\item \textbf{Seitenruder neutral stellen}
\item \textbf{Vorsichtig abfangen}
\end{enumerate}


\begin{color}{white}
\underline{Wichtiger Hinweis}\\
Der Höhenverlust kann beim Ausleiten bis zu $200m$ betragen!
\end{color}

\section{Beenden des Spiralsturzes}
Beim Trudeln wurde in keinen der durchgeführten Erprobungsszenarien eine Neigung zum Spiralsturz erkennbar.\\
Sollte sich trotzdem ein Spiralsturz einstellen, kann man ihn mit folgenden Steuereingaben ausleiten:
\begin{enumerate}
\item \textbf{Motor stoppen}
\item \textbf{Quer- und Seitenruder in Gegendrehrichtung}
\item \textbf{Vorsichtig Fahrt abbauen}
\end{enumerate}

\begin{color}{white}
\underline{Warnung}\\
Beim Abfangen sind die zulässigen Ruder- und Klappenausschläge zu den erreichten Geschwindigkeiten zu beachten.
\end{color}

\section{Notlandungen}
\subsection{Notlandung mit eingezogenem Fahrwerk}
Notlandung immer mit ausgefahrenem Fahrwerk, da der Pilot und die Flugzeugstruktur durch die Arbeitsaufnahme des gefederten Fahrwerks erheblich besser geschützt sind, als nur durch die Rumpfschale.\\
\newline
Lässt sich das Fahrwerk nicht ordnungsgemäß ausfahren, dann ist das Flugzeug in \textbf{Landestellung L der Wölbklappen} und mit \textbf{eingefahrenen Bremsklappen} in einem \textbf{flachen Winkel mit Mindestfahrt} aufzusetzen, um ein Durchsacken zu vermeiden.\\
Nach der Landung sollte eine gründliche Kontrolle der Flugzeugstruktur erfolgen.

\subsection{Notlandung auf dem Wasser}
Aus den bei Notlandungen auf Wasser gemachten Erfahrungen muss mit der Möglichkeit gerechnet werden, dass das gesamte Cockpit unter Wasser gedrückt wird. Bei Wassertiefen $>2m$ sind die Insassen in höchster Gefahr! Deshalb sollte die Notwasserung nur als letzter Ausweg gewählt werden.\\
\newline
Folgendes Vorgehen wird bei einer Notwasserung empfohlen:
\begin{itemize}
\item \textbf{Fahrwerk ausfahren}
\item \textbf{Fallschirmgurte öffnen}
\item \textbf{Aufsetzen mit ausgefahrenem Fahrwerk und möglichst geringer Geschwindigkeit}
\item \textbf{Das Cockpit sollte durch die Notfenster geflutet werden, um gegen den Wasserdruck die große Haube öffnen zu können}
\item \textbf{Nach dem Eintauchen Gurtzeug und Fallschirm ablegen}
\end{itemize}

\subsection{Drehlandung ("`Ringelpietz"')}
Wenn abzusehen ist, dass ein Landefeld von der Länge her nicht ausreicht, dann ist spätestens $50m$ vor Ende des Landefeldes eine gesteuerte Drehlandung einzuleiten:
\begin{enumerate}
\item \textbf{Flügel zur Ausweichrichtung hin auf den Boden steuern}
\item \textbf{Wenn möglich in den Gegenwind drehen}
\item \textbf{Gleichzeitig durch Nachdrücken den Sporn entlasten und durch gegensinniges Seitenruder der Torsion der Rumpfröhre entgegenwirken.}
\end{enumerate}


\section{Flug im Bereich von Gewittern}
Durch Blitzschlag sind wiederholt Kohlenstofffaserstrukturen zerstört worden. Flüge und besonders Windenschlepps im Bereich von Gewittern sind daher unbedingt zu vermeiden, da in wichtigen Strukturen der B13 Kohlenstofffasern verwendet werden.\\
Wenn der Verdacht auf Blitzschlag besteht oder ein solcher erfolgt ist, sollte die Fahrt auf \textbf{unter $V_A=160\frac{km}{h}$} reduziert werden. Die \textbf{Ruderwirksamkeit} ist zu \textbf{überprüfen} (Gefahr des Verschweißens der Rudergelenke) und \textbf{elektrische Systeme} sind \textbf{auszuschalten} um Kabelbrand zu vermeiden.
Zusätzlich sind der \textbf{Propeller einzufahren} und die \textbf{Nasenklappen zu schließen}, um das Eindringen von Wasser in den Motorraum zu verhindern.



\section{Flug bei Regen}
Flüge in starkem Regen und Gewittern sind zu vermeiden. Es wird empfohlen den Propeller einzufahren und die Nasenklappen zu schließen, um das Eindringen von Wasser in den Motorraum zu verhindern.
Wenn nötig ist ein Flug in leichtem Regen mit laufendem Motor möglich. Es sollte allerdings mit niedriger Leitungseinstellung geflogen werden, die ausreichend für den Horizontalflug ist, um Beschädigungen der Propellerblätter zu vermeiden. Bei starkem Regen muss der Motorbetrieb eingestellt werden. \\

Bei Regen verschlechtern sich die Flugleistungen. Es muss mit verstärktem Eigensinken und einer erhöhten Mindestfahrt gerechnet werden. Die Geschwindigkeit im Landeanflug sollte daher mindestens um $10 \frac{km}{h}$ erhöht werden. 

\section{Flug bei Vereisungsbedingungen}
Bei Vereisungsgefahr Gängigkeit der Ruder und Klappen durch ständiges Bewegen aufrechterhalten.
\newpage
\section{Triebwerksausfall}
\subsection{Motor startet nicht}
Falls der Motor nicht startet, muss der Flug im reinen Segelflug fortgesetzt werden. \\

\begin{color}{white}
\large{\underline{Anmerkung}}\\
Überprüfen, ob der Leistungsschalter eingeschaltet ist. Die Erinnerung (auf der FCU) “Check Power Switch” sollte ab einer bestimmten Leistungseinstellung erscheinen.
\end{color}

\subsection{Leistungsverlust während des Fluges}

Bei einem Leistungsverlust während des Fluges Steuerknüppel vorsichtig nach vorne drücken, um die gewünschte Fluggeschwindigkeit beizubehalten! Anschließend wie folgt verfahren:

\begin{enumerate}
\item Überprüfen, ob der Leistungsschalter unbeabsichtigt ausgeschaltet wurde! \\

In diesem Fall den Leistungsschalter wieder einschalten und die Leistung mit dem
Leistungsdrehregler anpassen.\\
\item Trifft Punkt 1 nicht zu, folgendermaßen fortfahren:\\
\begin{enumerate}
\item Zuerst den Leistungsschalter, dann die FCU ausschalten. \\
\item Schalten Sie die FCU wieder ein und überprüfen Sie, ob sich etwas ungewöhnlich verhält.\\

Wenn alles in Ordnung ist, Leistungsschalter wieder einschalten und Motor starten.
Wenn der Motor startet und sich unter Last ungewöhnlich verhält:
\begin{enumerate}
\item Drehenden Propeller mit der elektrischen Bremse anhalten.
\item Nachdem der Propeller gestoppt ist, zuerst den Leistungsschalter und dann die FCU ausschalten.
\end{enumerate}

Sollte der Propeller nicht gestoppt werden können, kann der Propeller mit dem Noteinfahrmechanismus abgebremst und eingefahren werden.
\end{enumerate}
\end{enumerate}

Noteinfahrmechanismus:
\begin{enumerate}
\item Engine Swich ausschalten
\item Sicherheitspin aus Schlittenhauptschalter herausziehen
\item Fluggeschwindigkeit verringern auf 80 km/h soweit möglich
\item Schlittenhauptschalter wiederholt kurzzeitig nach hinten drücken. Dies aktiviert den Schlittenmotor und zieht den Propeller gegen den Bremsgummiring
\item Sobald der Propeller abgebremst und eingeklappt ist, kann das Schlittensystem wieder normal in Betrieb genommen werden und der Propeller normal eingefahren werden.
\end{enumerate}

\begin{color}{white}
\large{\underline{Warnung}}\\
Der Propeller kann beim Noteinfahren beschädigt werden! Verwendung des Antriebssystems erst nach gründlicher Inspektion aller Bauteile.
\end{color}\\

Sollte dieser Noteinfahrmechanismus nicht funktionieren, muss mit drehendem Propeller gelandet werden. In diesem Fall muss bei der Landung vorsichtig und gleichzeitig auf beiden Rädern aufgesetzt werden (Zweipunktlandung), um eine Beschädigung des Propellers zu verhindern.\\

\begin{color}{white}
\large{\underline{Anmerkung}}\\
Eine Graspiste in gutem Zustand (ohne Schlaglocher oder Ähnlichem) ist einer Asphaltpiste vorzuziehen.
\end{color}\\

\begin{color}{white}
\large{\underline{Warnung}}\\
Landungen in hohem Bewuchs sind zu vermeiden.
\end{color}\\

\begin{color}{white}
\large{\underline{Anmerkung}}\\
Der Verlust der Gleitleistung durch den drehenden Propeller ist gering. Daher besteht bei ausreichend Höhe genug Zeit ein geeignetes Landefeld zu wählen.
\end{color}\\

Bitte lesen Sie das \textbf{FES FCU INSTRUMENTENHANDBUCH} für das Verhalten und die notwendigen Verfahren beim Erscheinen von bestimmten Nachrichten und Aufblinken von LED-Leuchten.

\section{Brand}
\subsection{Brand am Boden}

\begin{itemize}
\item Leistungsschalter ausschalten und alle Instrumente, sowie den Hauptschalter ausschalten
\item Cockpit verlassen
\item Brand löschen
\end{itemize}
\newpage
\subsection{Brand während des Fluges}
\begin{itemize}
\item Motor sofort abstellen
\item Leistungsschalter ausschalten und vordere Lüftung, falls noch nicht geöffnet, öffnen
\item Seitliches Haubenfenster offnen
\item So schnell wie möglich landen (oder gegebenenfalls einen Notausstieg in Betracht ziehen)
\item Nach der Landung Feuer löschen
\end{itemize}

\section{Sonstige Notfälle}

\subsection{Verlust der 12V Spannungsversorgung während des Fluges}

Segelflug:\\
Beim Ausfall der elektrischen Instrumente (Funkgerät, Bordrechner, FCU etc.)
wahrend des Segelfluges, muss der Flug im reinen Segelflug fortgesetzt werden. In
diesem Fall kann der Propeller nicht ausgefahren werden und der Motor nicht gestartet werden.\\

Ist die FCU nicht vom Ausfall betroffen, kann der Motorstart bei Bedarf versucht werden.\\

Motorflug:\\
Beim Ausfall der FCU während des Motorfluges, fällt auch der Motor aus. Ein Stoppen des drehenden Propellers durch die Noteinfahrfunktion möglich.\\

Fallen die Instrumente nur teilweise aus und ist die Funktion von FCU und Motor nicht beeinträchtigt, kann der Motor weiter betrieben werden.
