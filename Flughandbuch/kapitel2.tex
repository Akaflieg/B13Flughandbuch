\deftripstyle{Flughandbuch}[.5pt][.5pt]{\pagemark}{}{\headmark}{Flug- und Betriebshandbuch B13}{}{LBA-anerkannt 05.2013}
\pagestyle{Flughandbuch}
\renewcommand*\chapterpagestyle{Flughandbuch}


\chapter{Betriebsgrenzen und Angaben}

\section{Einführung}
Der vorliegende Abschnitt beinhaltet Betriebsgrenzen, Instrumentenmarkierungen und die Hinweisschilder, die für den sicheren Betrieb des Motorseglers B13, seiner werksseitig vorgesehenen Systeme und Anlagen und der werksseitig vorgesehenen Ausrüstung notwendig sind. Die in diesem Abschnitt angegebenen Betriebsgrenzen sind vom Luftfahrt-Bundesamt zugelassen.

\section{Fluggeschwindigkeiten}

Die Fluggeschwindigkeitsgrenzen und ihre Bedeutung für den Betrieb sind nachfolgend aufgeführt:

\begin{longtable}{|c|p{3cm}|c|p{4cm}|}
\hline
& Geschwindigkeit & IAS [$\frac{km}{h}$] & Anmerkungen \\
\hline
$V_{NE}$ & Zulässige Höchstgeschwindigkeit bei ruhigem Wetter & $220$ & Diese Geschwindigkeit darf nicht überschritten werden und der Ruderausschlag darf nicht mehr als $\frac{1}{3}$ betragen\\
\hline
$V_{RA}$ & Zulässige Höchstgeschwindigkeit in starker Turbulenz & $160$ & Diese Geschwindigkeit darf bei starker Turbulenz nicht überschritten werden. (Starke Turbulenz herrscht vor in Leewellen-Rotoren, Gewitterwolken, usw.) \\
\hline
$V_A$ & Manöver-\newline geschwindigkeit & $160$ & Oberhalb dieser Geschwindigkeit dürfen keine vollen oder abrupten Ruderausschläge ausgeführt werden, da die Flugzeugstruktur dabei überlastet werden könnte.\\
\hline
$V_{FE}$ & Zulässige Höchstgeschwindigkeit für das Betätigen der Flügelklappen
&   & Diese Geschwindigkeit darf bei der angegebenen Flügelklappenstellung nicht überschritten werden.\\
& +2, +1 & 160 & \\
& Landestellung L & 130 & \\
\hline
$V_W$ & Zulässige Höchstgeschwindigkeit für den Windenschlepp & $120$ & Diese Geschwindigkeit darf während des WInden- oder Kraftfahrzeugschlepps nicht überschritten werden.\\
\hline
$V_T$ & Zulässige Höchstgeschwindigkeit für den Flugzeugschlepp & $160$ & Diese Geschwindigkeit darf während des Flugzeugschlepps nicht überschritten werden.\\
\hline
$V_{LO}$ & Zulässige Höchstgeschwindigkeit zum Betätigen des Fahrwerks & $160$ & Über dieser Geschwindigkeit darf das Fahrwerk nicht ein- oder ausgefahren werden.\\
\hline
$V_{PO,max}$ & Zulässige Höchstgeschwindigkeit für das Ein- und Ausfahren des Triebwerks & -- & Triebwerk nicht eingebaut\\
\hline
$V_{PO,max}$ & Zulässige Mindestgeschwindigkeit für das Ein- und Ausfahren des Triebwerks & -- & Triebwerk nicht eingebaut\\
\hline
\end{longtable}
\newpage
\section{Anzeigefehler in der Fahrtmesseranlage}
Die folgenden Angaben sind als die berichtigten Fluggeschwindigkeiten (VCAS) über der angezeigten Fluggeschwindigkeit (VIAS) dargestellt. Es wurde dabei ein Instrumentenfehler gleich Null angenommen. Die Darstellung erfasst weiterhin alle Flügelklappenstellungen und deckt den entsprechenden Geschwindigkeitsbereich ab.\\
\newline
Die Druckentnahme erfolgt durch eine Kombidüse an der Nase vom Seitenleitwerk.\\
\newline
\includegraphics[width=.9\textwidth]{fahrtmesserkalibrierung.png}
\newline
Der Fehler der Fahrtmesseranlage beträgt nicht mehr als $8 \frac{km}{h}$ bzw. $5\%$ und erfüllt damit die Anforderungen der JAR 22 (siehe JAR 22.1323).

\section{Überziehgeschwindigkeiten}

Die folgenden Überziehgeschwindigkeiten wurden bei einem Abfluggewicht von $820kg$ und vorderster Schwerpunktlage ermittelt:

\begin{tabular}{l l l}
WK $-2$ & & \\
& Geradeaus: & $78\frac{km}{h}$\\
& $5^{\circ}$ schiebend: & $78\frac{km}{h}$\\
& $45^{\circ}$-Kurve: & $95\frac{km}{h}$\\
%& &  \\
\hline
& &  \\
WK $-1$ & & \\
& Geradeaus: & $79\frac{km}{h}$\\
& $5^{\circ}$ schiebend: & $79\frac{km}{h}$\\
& $45^{\circ}$-Kurve: & $88\frac{km}{h}$\\
%& &  \\
\hline
& &  \\
WK $0$ & & \\
& Geradeaus: & $79\frac{km}{h}$\\
& $5^{\circ}$ schiebend: & $79\frac{km}{h}$\\
& $45^{\circ}$-Kurve: & $90\frac{km}{h}$\\
%& &  \\
\hline
& &  \\
WK $+1$ & & \\
& Geradeaus: & $77\frac{km}{h}$\\
& $5^{\circ}$ schiebend: & $77\frac{km}{h}$\\
& $45^{\circ}$-Kurve: & $89\frac{km}{h}$\\
%& &  \\
\hline
 & & \\
WK $+2$ & & \\
& Geradeaus: & $76\frac{km}{h}$\\
& $5^{\circ}$ schiebend: & $76\frac{km}{h}$\\
& $45^{\circ}$-Kurve: & $85\frac{km}{h}$\\
%& &  \\
\hline
& &  \\
WK L& & \\
& Geradeaus: & $75\frac{km}{h}$\\
& $5^{\circ}$ schiebend: & $75\frac{km}{h}$\\
& $45^{\circ}$-Kurve: & $88\frac{km}{h}$\\
& geradeaus, Bremsklappe + Fahrwerk aus: & $78\frac{km}{h}$ \\
\end{tabular}
\section{Fahrtmessermarkierungen}
Die folgende Tabelle nennt die Fahrtmessermarkierungen und die Bedeutung der Farben:\\

\begin{tabular}{|m{1,8cm}|m{1,5cm}|m{6cm}|}
\hline
Markierung & IAS [$\frac{km}{h}$] & Bedeutung \\
\hline
Weißer Bogen & $80-160$ & Betriebsbereich für positive Klappenausschläge (Untere Grenze ist die Geschwindigkeit $1,1 V_{S0}$ bei Höchstmasse in Landekonfiguration. Obere Grenze ist die zulässige Höchstgeschwindigkeit mit positivem Klappenausschlag.)\\
\hline
\begin{color}{forestgreen} Grüner Bogen \end{color} & \begin{color}{forestgreen} $80-160$ \end{color} & \begin{color}{forestgreen} Normaler Betriebsbereich (Untere Grenze ist die Geschwindigkeit $1,1 V_{S1}$ bei Höchstmasse und vorderster Schwerpunktlage und Flügelklappen in der Neutralstellung; obere Grenze ist die zulässige Höchstgeschwindigkeit in starker Turbulenz) \end{color} \\ 
\hline
\begin{color}{myyellow} Gelber Bogen \end{color} & \begin{color}{myyellow} $160-220$ \end{color} & \begin{color}{myyellow} In diesem Bereich darf bei starker Turbulenz nicht gefolgen werden und Manöver dürfen nur mit Vorsicht durchgeführt werden. \end{color}\\
\hline
\begin{color}{red} Roter Strich \end{color} & \begin{color}{red} $220$ \end{color} & \begin{color}{red} Zulässige Höchstgeschwindigkeit für alle Betriebsarten \end{color}\\
\hline
\begin{color}{myyellow} Gelbes Dreieck \end{color} & \begin{color}{myyellow} $100$ \end{color} & \begin{color}{myyellow} Anfluggeschwindigkeit bei Höchstmasse \end{color}\\
\hline

\end{tabular}
\newpage
\section{Triebwerk}
nicht eingebaut
\newpage
\section{Triebwerksinstrumente}
nicht eingebaut
\newpage
\section{Masse (Gewicht)}
\begin{tabular}{l l}
Leermasse & s. Wägebericht\\
Höchstzulässige Abflugmasse & $820kg$\\
Höchstzulässige Masse nichttragender Teile & $500kg$ \\
Höchstmasse im Gepäckraum & $10kg$ \\

\end{tabular}

\section{Schwerpunkt}
\begin{tabular}{l l}
Flugzeuglage & Keil $1000:28$ auf der Rumpfoberseite\\
Bezugsebene (BE) & Flügelvorderkante an der Wurzelrippe \\
Größte Vorlage & $245,3mm$ hinter BE\\
Größte Rücklage & $428,6mm$ hinter BE\\
\end{tabular}\\
\newline
\textbf{Hebelarme}\\
\begin{tabular}{m{6,5cm} m{3cm}}
Piloten & $x_P=-445mm$\\
Gepäckfach & $x_G=200mm$\\

\end{tabular}\\
\newline
Die Bezugsebene ist die Vorderkante der Wurzelrippe.

\subsection{Wägebericht}

\begin{tiny}
\begin{tabular}{|m{1,8cm}|m{1,8cm}|m{2cm}|m{1,5cm}|m{1,5cm}|}
\hline
Datum & Leermasse [$kg$] & Leermassen- schwerpunkt [$mm$]  & Maximale Zuladung [$kg$] & Unterschrift\\

\hline
& & & &\\
14.03.12 & 579 & 552,9 & 221 & Hofmann\\
& & & &\\
\hline
& & & &\\
& & & &\\
& & & &\\
\hline
& & & &\\
& & & &\\
& & & &\\
\hline
& & & &\\
& & & &\\
& & & &\\
\hline
& & & &\\
& & & &\\
& & & &\\
\hline

\end{tabular}
\end{tiny}

%\newline
%Weitere Hinweise zur Schwerpunktlage und dem Beladeplan sind dem Kapitel 6 "`Masse und Schwerpunktlage"' zu entnehmen.

\section{Zugelassene Manöver}
Der Motorsegler B13 ist für den normalen Segelflug (Lufttüchtigkeitsgruppe "`Utility"') zugelassen.\\
\newline
\begin{color}{red}
\textbf{Kunstflug ist nicht zulässig}
\end{color}

\section{Manöverlastvielfache}

Folgende Lastvielfache dürfen beim Abfangen nicht überschritten werden.\\
\newline
\begin{tabular}{|l|c|c|}
\hline
& positiv & negativ \\
\hline
Bei Manövergeschwindigkeit $V_A=160\frac{km}{h}$ & $+5,3$ & $-2,65$\\
\hline
Bei Höchstgeschwindigkeit $V_{NE}=220^\frac{km}{h}$ & $+4,0$ & $-1,5$ \\
\hline
Bei ausgefahrenen Bremsklappen und $V_{NE}$ & $+3,5$ & $0$\\
\hline
\end{tabular}

\section{Flugbesatzung}
Die B13 kann einsitzig oder doppelsitzig geflogen werden. Der verantwortliche Luftfahrzeugführer kann auf der linken oder rechten Seite sitzen. Es wird empfohlen, die Platzrunde und insbesondere bodennahe Kurven (z.B. bei Seilriss) in Richtung des fliegenden Luftfahrzeugführes auszuführen, da ansonsten mit Sichtbeeinträchtigungen gerechnet werden muss.

\section{Betriebsarten}
Mit der B13 dürfen Flüge nach Sichtflugregeln (VFR) bei Tag durchgeführt werden.\\
\newline
\begin{color}{red} Kunstflug und Wolkenflug sind nicht zulässig
\end{color}

\section{Mindestausrüstung}
Zur Mindestausrüstung für den Normalbetrieb gehören:\\
\begin{itemize}
\item Fahrtmesser (bis $300\frac{km}{h}$ mit Farbmarkierungen nach Abschnitt 2.3)
\item Höhenmesser
\item Variometer
\item Magnetkompass
\item 2 Anschnallgurte (vierteilig, symmetrisch)
\item Flug- und Betriebshandbuch
\item Daten- und Hinweisschilder
\item 2 automatische oder manuelle Fallschirme
\end{itemize}

\section{Flugzeugschlepp und Windenschlepp}

\textbf{Flugzeugschlepp}\\
Die maximal zulässige Schleppgeschwindigkeit beträgt $V_T=160\frac{km}{h}$.\\
Es wurden Seillängen zwischen $30m$ und $60m$ erprobt.\\
Die Sollbruchstellen des Schleppseils sollten eine Bruchlast von $1000daN$ (schwarz) erreichen.\\
Für den Flugzeugschlepp wird die Schwerpunktkupplung an der Rumpfunterseite verwendet.\\

\textbf{Windenschlepp}\\
Die maximal zulässige Schleppgeschwindigkeit beträgt $V_W=120\frac{km}{h}$, $100\frac{km}{h}$ sollte nicht unterschritten werden.\\
Die Sollbruchstellen des Windenseils sollten eine Bruchlast von $1000daN$ (schwarz) haben.\\
Für den Windenstart wird die Schwerpunktkupplung an der Rumpfunterseite verwendet.
\newpage
\section{Hinweisschilder}

\begin{figure}[h]
\begin{center}
\includegraphics[width=.45\textwidth]{bilder/startcheck.pdf}
\caption*{Startcheck}
\end{center}
\end{figure}

\begin{figure}[h]
\begin{center}
\includegraphics[width=.9\textwidth]{bilder/vvz.pdf}
\caption*{Permit To Fly}
\end{center}
\end{figure}

\begin{figure}[h]
\begin{center}
\includegraphics[width=.45\textwidth]{bilder/datenschild.pdf}
\caption*{Datenschild}
\end{center}
\end{figure}

\begin{figure}[h]
\begin{center}
\includegraphics[width=.15\textwidth]{bilder/notabwurf.pdf}\\
\caption*{Haubennotabwurf am Instrumentenbrett}
\end{center}
\end{figure}

\begin{figure}[h]
\begin{center}
\includegraphics[width=.45\textwidth]{bilder/bk.pdf}
\caption*{Bremsklappen, blauer Griff jeweils links}
\end{center}
\end{figure}


\begin{figure}[h]
\begin{center}
\includegraphics[width=.45\textwidth]{bilder/fahrwerk.pdf}
\caption*{Fahrwerk, silberner Hebel in der Mitte}
\end{center}
\end{figure}

\begin{figure}[h]
\begin{center}
\includegraphics[width=.45\textwidth]{bilder/trimmung.pdf}
\caption*{Trimmung, grüner Hebel in der Mitte}
\end{center}
\end{figure}

\begin{figure}[h]
\begin{center}
\includegraphics[width=.15\textwidth]{bilder/lueftung.pdf}
\caption*{Lüftungsbetätigung, Knopf links und rechts an Cockpitwand}
\end{center}
\end{figure}

\begin{figure}[h]
\begin{center}
\includegraphics[width=.15\textwidth]{bilder/kupplung.pdf}
\caption*{Schleppkupplung, gelber Griff jeweils links neben Steuerknüppel}
\end{center}
\end{figure}

\begin{figure}[h]
\begin{center}
\includegraphics[width=.15\textwidth]{bilder/pedale.pdf}
\caption*{Pedalverstellung, weißer Griff rechts neben Steuerknüppel}
\end{center}
\end{figure}

\begin{figure}[ht]
\begin{center}
\includegraphics[width=.45\textwidth]{bilder/wk.pdf}
\caption*{Wölbklappenhebel, schwarzer Griff, jeweils links}
\end{center}
\end{figure}

